%File: formatting-instruction.tex
\documentclass[letterpaper]{article}
\usepackage{flairs}
\usepackage{times}
\usepackage{helvet}
\usepackage{courier}
\usepackage{graphicx}
\usepackage{setspace}
\frenchspacing
\usepackage{verbatim}

\setlength{\pdfpagewidth}{8.5in}
\setlength{\pdfpageheight}{11in}

\newcommand{\mytilde}{\raisebox{0.4ex}{\texttildelow}}
%----Making things more compact
\newcommand{\smalltt}[1]{{\small \tt #1}}
\newenvironment{packed_itemize}{
\vspace*{-0.2em}
\begin{itemize}
\setlength{\partopsep}{0pt}
\setlength{\itemsep}{1pt}
\setlength{\parskip}{0pt}
\setlength{\parsep}{0pt}
}{\end{itemize}}
\newenvironment{packed_enumerate}{
\vspace*{-0.2em}
\begin{enumerate}
\setlength{\partopsep}{0pt}
\setlength{\itemsep}{1pt}
\setlength{\parskip}{0pt}
\setlength{\parsep}{0pt}
}{\end{enumerate}}
\renewcommand{\textfraction}{0.07}
\renewcommand{\topfraction}{0.9}
\renewcommand{\bottomfraction}{0.9}
\renewcommand{\floatpagefraction}{0.66}
\setlength{\floatsep}{2.0pt plus 2.0pt minus 2.0pt}
\setlength{\textfloatsep}{10.0pt plus 2.0pt minus 0.0pt}

% \pdfinfo{
% /Title (ShZZaM: An LLM+ATP Natural Language to Logic Translator)
% /Author (Geoff Sutcliffe, Danial Haroon)}
% \setcounter{secnumdepth}{2}  
% \begin{document}
% \title{ShZZaM: An LLM+ATP Natural Language to Logic Translator}
% \author{Geoff Sutcliffe, Danial Haroon\\
% University of Miami\\
% Coral Gables, USA
% }
\pdfinfo{
/Title (ShZZaM: An LLM+ATP Natural Language to Logic Translator)
/Author (First1 Last1, First2 Last2)}
\setcounter{secnumdepth}{2}  
\begin{document}
\title{ShZZaM: An LLM+ATP Natural Language to Logic Translator}
\author{First1 Last1, First2 Last2\\
Affiliation\\
City, Country
}
\maketitle
\begin{abstract}
\begin{quote}
This paper describes the ShZZaM tool that uses Large Language Models (LLMs) and Automated
Theorem Proving (ATP) tools to translate natural language to typed first-order logic in
the TFF syntax of the TPTP World.
MORE
\end{quote}
\end{abstract}
%--------------------------------------------------------------------------------------------------
Large Language Models (LLMs)~\cite{PR+25} have shown themselves to be useful in a broad range
of applications~\cite{MK+25}.
However, it is well known that LLMs make mistakes~\cite{HY+25}, and this is acknowledged on 
LLMs' web interfaces, e.g., ChatGPT admits "ChatGPT can make mistakes. Check important info".
In the face of such unreliability, the results from LLMs in mission-critical applications require
verification.
One approach is to translate the LLM input and output to a logical form that can be checked
using Automated Theorem Proving (ATP) tools, e.g.,~\cite{YS+25,CL+25}.\footnote{%
For a more comprehensive survey, just ask your favourite LLM to ``show some research on how LLMs 
make mistakes, and the need for symbolic checking of LLM output''.}
A key step in this verification pipeline is the faithful translation of the natural language to
an appropriate logical form.
This task is difficult due to the ambiguous nature of natural language statements, especially
informally expressed statements.
Work in this area includes LINC~\cite{OG+23}, FOLIO~\cite{HS+24}, and LINA~\cite{LL+24-arXiv}.
This paper makes another contribution in this area, taking a new interactive approach to the
translation process, zigzagging (hence the 'ZZ' in the tool name) between natural language and 
logic until convergence is achieved.
A key feature of ShZZaM is its use of LLMs and Automated Theorem Proving (ATP) tools, which
complement each other in the translation steps.

Figure~\ref{Process} shows the overall process implemented of ShZZaM.
Starting with the natural language, a combination of LLMs and ATP tools make a first translation
(step 1 - a ``Zig'') to the typed first-order logic in the TFF syntax~\cite{SS+12,BP13-TFF1} of 
the TPTP World~\cite{Sut24}.
An LLM is then used to translate the logic back to natural language (step 2 - a ``Zag'').
An LLM is then used to judge (step 3) whether or not the new and previous natural language 
statements have the same meaning, at a given level of required similarity.
If they do, the logic is accepted as the translation.
If they do not, another zigzag is performed, continuing until the natural language pairs 
converge to the required level of similarity (or a limit is reached).

LLMs know TFF - translation works.
LLMs know language - similarity works.


\begin{figure*}[bt]
\centering
\includegraphics[width=\textwidth]{Process.pdf}
\caption{ShZZaM process}
\label{Process}
\end{figure*}

Automated Theorem Proving (ATP) deals with the task of proving theorems from axioms – the
derivation of conclusions that follow inevitably from known facts~\cite{RV01-HAR}.
The converse task of disproving conjectures is another facet of interest~\cite{CS03,BN10-ITP}.
The axioms and conjectures are written in an appropriately expressive logic, and the solutions
(proofs and models) are often similarly written in logic~\cite{Sut23-IGPL}.
The TPTP World~\cite{Sut17} is the well established infrastructure that supports research, 
development, and deployment of ATP systems.
The TPTP World includes the TPTP problem library~\cite{Sut17}, 
the TSTP solution library~\cite{Sut10}, 
standards for writing ATP problems and reporting ATP solutions~\cite{SS+06,Sut08-KEAPPA}, 
tools and services for processing ATP problems and solutions~\cite{Sut10}, 
and it supports the CADE ATP System Competition (CASC)~\cite{Sut16}.
% Various parts of the TPTP World have been used in a range of applications, in both academia and 
% industry.
The web page \smalltt{tptp.org} provides access to all components.

The TPTP language~\cite{Sut23-IGPL} is one of the keys to the success of the TPTP World.
The TPTP language is used for writing both problems and solutions, which enables convenient 
communication between ATP systems and tools.
% Originally the TPTP World supported only first-order clause normal form (CNF)
% \cite{SS98-JAR}.
% Over the years full first-order form (FOF)
% \cite{Sut09}, 
% typed first-order form (TFF)
% \cite{SS+12,BP13-TFF1}, 
% typed extended first-order form (TXF)
% \cite{SK18}, 
% typed higher-order form (THF)
% \cite{SB10,KSR16}, 
% and non-classical forms (NTF)~\cite{SF+22} have been added.
Problems and solutions are built from {\em annotated formulae} of the form:
\begin{center}
{\em language}{\tt (}{\em name}{\tt ,}
{\em role}{\tt ,}
{\em formula}{\tt ,}
{\em source}{\tt ,}
{\em useful\_info}{\tt )}
\end{center}
The {\em language}s supported are \smalltt{cnf} (clause normal form), \smalltt{fof}
(first-order form), \smalltt{tff} (typed first-order form), and \smalltt{thf}
(typed higher-order form).
The {\em role}, e.g., \smalltt{axiom}, \smalltt{lemma}, \smalltt{conjecture}, defines the 
use of the formula.
In a {\em formula}, terms and atoms follow Prolog conventions -- functions and predicates start 
with a lowercase letter and variables start with an uppercase letter.
The language also supports interpreted symbols,
% that either start with a {\tt \$}, 
e.g., the truth constants \smalltt{\$true} and \smalltt{\$false}.
% or are composed of 
% non-alphabetic characters, e.g., integer/rational/real numbers such as 27, 43/92, -99.66.
The main logical connectives in the TPTP language are
% {\tt !>}, {\tt ?*}, {\tt @+}, {\tt @-}, 
{\tt !}, {\tt ?}, {\tt {\mytilde}}, {\tt |}, {\tt \&}, 
{\tt =>}, {\tt <=}, {\tt <=>}, and {\tt <{\mytilde}>}, for the mathematical connectives
% $\Pi$, $\Sigma$, choice (indefinite description), definite description,
$\forall$, $\exists$, $\neg$, $\vee$, $\wedge$, $\Rightarrow$, $\Leftarrow$, $\Leftrightarrow$, 
and $\oplus$ respectively.
Equality and inequality are expressed as the infix operators {\tt =} and {\tt !=}.
The {\em source} and {\em useful\_info} are optional.

%--------------------------------------------------------------------------------------------------
\section{Example Solutions and their Visualizations}
\label{Example}


%--------------------------------------------------------------------------------------------------
\section{Conclusion}
\label{Conclusion}

This paper describes the derivation and interpretation viewers in the TPTP World: 
the Interactive Derivation Viewer (IDV), the Interactive Tableau Viewer (ITV), the Interactive 
Interpretation Viewer (IIV), and the Interactive Kripke Viewer (IKV).
Users and developers of ATP systems are able to examine their ATP solutions in an interactive
graphical environment, providing insights into features of the solutions.
The viewers are freely accessible through SystemOnTPTP.

%--------------------------------------------------------------------------------------------------
\bibliographystyle{flairs}
\bibliography{Bibliography.bib}
%--------------------------------------------------------------------------------------------------
\end{document}
